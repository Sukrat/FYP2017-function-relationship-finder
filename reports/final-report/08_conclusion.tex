\section{Conclusion}

In this project, a successful software was created which enabled the ability to understand the relation between input parameter variable and the output variable. The software was run on experimental data set where the input parameters and the output variable corresponds to the performance events and the energy consumption. The software enabled us to cluster the data with various tolerance and analyse the effect of tolerance. This also helps in understanding and finding the optimum tolerance in the dataset where the functional relation is linear the most. It can be used for validating the linear models already in place and understand the relationship between variables on a much higher level.

The project has certain limitation such as the presence of a non-influential performance event or parameter could lead inaccurate result. The software works well with noise as the first step of the existence of functional relation could stop the noise from the analysis. But the addition of noise variable would corrupt the results significantly. Another limitation of the project is that it can only verify linear models and can only analyse linear relation.

\section{Future work}

More work is needed in the clustering algorithm because the Euclidean distance is not always the right choice for finding the difference between two points. As per\cite{aggarwal2001surprising}, the Euclidean distance measure does not work accurately because of the effect of the curse of high dimensionality.

The regression analysis right now is only linear. We would want to be able to analyse and validate other models such as \(\log\), polynomial etc. Each cluster must go through different regression analysis and pick the model which best fits the dataset.