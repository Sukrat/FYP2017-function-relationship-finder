Many designers are increasingly utilizing dynamic hardware adaptations to improve performance while limiting the power consumption. Some are using software to decrease power usage for e.g. putting the system in sleep mode when it's in idle state. The main goal remains the same, which is to extract maximum performance while minimizing the temperature and power. Whereas, we want to study and examine the relationship affecting the consumption and then analyse the result to minimize or predict the consumption of energy.

\section{Energy consumption and Performance Events}

First let's look at energy consumption. Energy consumption is the power (Usually in watts) consumed by a system. This system could be the processor/CPU, memory, disk, I/O (Input/Output) system, chipset or the whole computer system itself. So, one can take any of the peripherals and read the power consumption for various performance events. Then analyse if there exists a functional dependence to begin with. If the system is not able to disprove the dataset, one could then try to find a function which could help understand relation between each event and predict for any given system. Reading of these performance events can be during a idle state as well as running certain computations.

Now let's look at what are performance events, performance events can be any event which can affect the consumption of energy in some way. Selection of performance events is quite challenging. A simple example would be the effect of cache misses in the processor. For a typical processor, the highest level of cache would be L3 or L2 depending on the type of processor. Now for some transaction which could not be found in the highest level of cache (cache miss) would cause a cache block size access to the main memory. Thus, number of main memory access would be directly proportional to the cache misses. Since these memory access is off-chip, power is consumed in the memory controller and DRAM. Even though, the relation is not simple as it seems but we can see a strong casual relationship between the cache miss and the main memory power consumption~\cite{bircher2007complete}.

We can use number of other performance events like Instructions executed. As we know on each instruction being executed, more units of the system are on. Hence, power is consumed as opposed to when the processor is in its idle state~\cite{gilberto2005power}. 

Cache miss, TLB misses are also a good performance events as they seem to have a strong relationship between the power consumption as processor needs to handle memory page walks. Same can be said for Page faults where a program is not able to find mapped address in physical memory as it has not been loaded yet. This causes a trap which can result into number of situations, one of them which is to get the data from disk. In simple terms it is longer walk from cache miss. This walk to the disk and raising of exceptions would consume more energy by the disk as well as the CPU.

\section{Proofs}

The Proofs below are for different approaches that have been discussed to find the non-existence of a functional relations between energy consumption and number of performance events.

But first let's look at our dataset that will be provided.
We know that the data will be in the following format:

Let \(k\) be the number of parameters for the energy and \(n\) be the number of records in the dataset

\(E_1,\ x_{11},\ x_{12},\ \ldots x_{1k}\)\\
\(E_2,\ x_{21},\ x_{22},\ \ldots x_{2k}\)\\
\ldots\\
\(E_n,\ x_{n1},\ x_{n2},\ \ldots x_{nk}\)\\
where \(E_n\) is the dynamic energy for the nth tuple and \(x_{nk}\) corresponds to the \(k\)th performance event for \(n\)th record.

We will use mathematical definition of functional relationship to prove the approaches:

\textbf{Definition}: \textit{Given a dataset of pairs \((x_i, y_i)\) where \(i \in [1, n]\) of two variables \(x\) and \(y\), and the range \(X\) of \(x\), \(y\) is a function of \(x\) iff for each \(x_0 \in X\), there is exactly one value of \(y\), say \(y_0\), such that \((x_0, y_0)\) is in dataset.}~\cite{zembowicz1993testing}

\textbf{Prove:} We need to prove that finding atleast 2 equal performance events with different dynamic energies ensures that there exists no functional relationship in the dataset.

\textbf{Proof:}\\
Let us assume that there exists a functional relation such that:

\(f(x_{n1},\ x_{n2},\ \ldots x_{nk}) = E_n\)\\
where \(f\) is the functional relation for the dataset.

Our task is to find \(f(x_{i1},\ x_{i2},\ \ldots x_{ik}) = E_i\) and \(f(x_{j1},\ x_{j2},\ \ldots x_{jk}) = E_j\) 
where \(i \neq j\) and \(E_i \neq E_j\) and \((x_{i1},\ x_{i2},\ \ldots x_{ik}) = (x_{j1},\ x_{j2},\ \ldots x_{jk})\).

If such \(i\) and \(j\) exists. Then, we can conclude that the \(f\) is not a function by using the definition of a function as this assumed function has two images.

Which contradicts from our hypothesis stated above.
Hence by proof of contradiction we could say that \(f\) is not a function on the dataset.

Restating the above we can say dataset does not contain a functional relation.

\textbf{Prove:} Assuming that the dataset given has linear relationship and if we are able to find the constant for any one of the events. And if it does not apply to the other tuples of data that means linear relationship doesnot exist between the dataset.

\textbf{Proof:}\\
Let us assume that the energy consumption is a linear combination of the performance events.

\(f(x_{i1},\ x_{i2},\ \ldots x_{ik}) = E_i\)\\
\(f(x_{i1},\ x_{i2},\ \ldots x_{ik}) = (\alpha _1\times x_{i1}) + (\alpha_2\times x_{i2}) + \cdots + (\alpha_k\times x_{ik}) + \alpha_{k+1}\)\\
where \(\alpha_i\) are the constants in the linear combination of performance events and \(i\in[1\ldots k+1]\)

Lets find \(\alpha_i\) where \(i\in[1\ldots k+1]\)\\
To find this we will have to find atleast 3 records which have their parameter events equal \(x_{1},\ x_{2},\ \ldots x_{k}\) except \(x_{i}\) where this \(x\) values are value belonging to a row in the dataset.

If we found three records such as:

\(E_a,\ x_{a1},\ x_{a2}, \ldots x_{ak}\)\\
\(E_b,\ x_{b1},\ x_{b2}, \ldots x_{bk}\)\\
\(E_c,\ x_{c1},\ x_{c2}, \ldots x_{ck}\)\\
where the tuples \\
\((x_{a1}\ldots \ x_{a(i-1)},\ x_{a(i+1)}\ \ldots x_{ak})\), \((x_{b1}\ldots \ x_{b(i-1)},\ x_{b(i+1)}\ \ldots x_{bk})\) and \((x_{c1}\ldots \ x_{c(i-1)},\ x_{c(i+1)}\ \ldots x_{ck})\) are equal to each other, except \(x_{i}\) for some \(i\in[1\ldots k+1]\) where \(a, b, c \in [1, n]\) and \(a, b, c\) are not equal to each other.

Then

\(E_a - E_b\)\\
\(= f(x_{a1},\ x_{a2},\ \ldots x_{ak}) - f(x_{b1},\ x_{b2},\ \ldots x_{bk})\)\\
{\footnotesize \{ By our assumption that \(E\) is a linear combination of its parameters \} }\\
\(= ((\alpha _1\times x_{a1}) + \cdots + (\alpha_{k}\times x_{ak}) + \alpha_{k+1}) - ((\alpha _1\times x_{b1}) + \cdots + (\alpha_k\times x_{bk}) + \alpha_{k+1})\)\\
{\footnotesize \{ Gathering terms \}}\\
\(= \alpha_1 \times (x_{a1} - x_{b1}) + \ldots + \alpha_k \times (x_{ak} - x_{bk}) + (\alpha_{k+1} - \alpha_{k+1})\)\\
{\footnotesize \{ Since we know except \(x_{mi}\) and \(x_{ni}\) all are equal \}}\\
\(= \alpha_i \times (x_{ai} - x_{bi})\)

From the above we get:

\(\alpha_i = (E_a - E_b)/(x_{ai} - x_{bi})\)\\
where \((x_{ai} - x_{bi}) \neq 0\) as \(x_{ai} \neq x_{bi}\) by above during our finding phase.

Then we know that using the \(\alpha_i\) and applying to result to this equation \(E_a - E_c = \alpha_i \times (x_{ai} - x_{ci})\) must be true as well.

If this is false then \(\alpha_i\) is not a constant which contradicts our assumption that our \(E\) is linear combination of its parameter is false.

Hence using proof by contradiction we can say that the dataset is not linear combination of its parameters.

\textbf{Prove:} Keeping our assumption that the dataset is additive. Creating Pseudo records by simple addition with other records if results into a record whose energy consumption lies in the dataset but with different performance events shows that the dataset is not linear.

\textbf{Proof:}\\
Let us assume that there exists a linear function \(f\) such that:

\(f(x_{i1} + x_{j1},\ x_{i2}+ x_{j2},\ \ldots x_{ik} + x_{jk}) = E_i + E_j\)\\ and 
\(f(x_{i1},\ x_{i2},\ \ldots x_{ik}) + f(x_{j1},\ x_{j2},\ \ldots x_{jk}) = E_i + E_j\)\\
where \(\alpha_i\) are the constants in the linear combination of performance events and \(i\in[1\ldots k+1]\)

We know that if \(f\) is additive, hence new records can be generated via addition of records in the dataset.

Let \(V\) be the set of all dataset rows and dataset rows possible by combining various records in the dataset and pseudo records.

Now if we are able to find data record with \((x_{i1},\ x_{i2},\ \ldots x_{ik}) = (x_{j1},\ x_{j2},\ \ldots x_{jk})\) where \(i \neq j\) and \(E_i \neq E_j\) , where both records belong to the set \(V\).

Then, by the first proof, \(f\) is not an linear function. Which contradicts our assumption.

Hence using proof by contradiction we can say that the dataset does not contain a linear function.

\section{Applications}

As you can see from the proofs above, if any of the conditions above is satisfied then we are able to show that there does not exist any functional relationship between the events and the power consumption. If none of the conditions are satisfied then that shows that there might be an existence of a functional relation. However, it does not guarantee the existence of any form. Non-existence of a function and linear functions are validated. The reason for making a software like this verifies and gives the user the confidence. If data do not fit any functional hypothesis in a space, much time could be saved by preventing the unneeded search of the form of hypothesis as the software will only test the basic conditions that are not supposed to be there for a functional hypothesis.

The software is not restricted to the use of only on dataset which consists of performance events and power consumption. It is a general-purpose software which will for work for any kind of dataset in which user wants to know the existence of functional relation. The refuting of the claim of functional relation on dataset is the objective of the software.

We also know that the dataset provided is usually experimentally measured values which are not accurate. Every measuring device has some margin of error. The software will be flexible in the sense that the equality comparison of values in the approaches will always be done keeping in the margin of error provided.
