\section{Motivation}
Modern day technology develop under incredible speed in recent decade and the computing power growth rate is truly phenomenal and lasting impact can be felt and benefit us in many ways.\\
For example, back in the early 90s, the power factory generate electricity at a constant speed 24 hours a day but the demand varies throughout the day. Thus, surplus electricity generally goes to waste. Although some of them are sourced from nuclear power station which produces at constant rate throughout, however,  there are still large portion of electricity generated from fossil fuels, coal, petroleum, natural gas etc. control the supply of those source are practical. Thus, if we could discover the function relation between the demand of the power and the performance event (PMCs), that could enable us the ability to adjust the supply within certain network or distributed the power to some other network which power is more demanding.\\
Above is just only one field that possible usage of the functional relationship in data. In real world there are much more application that required such technique. Such as hybrid vehicle power management, power supply of auxiliary power units etc. There are many past work in this field but only few were focus on the relationship between dynamic energy consumption and PMCs. Hence, in this report we will try to observe the relationship between energy consumption and PMCs.  is visualized to see the nature of the functional relationship. And after that we will explore the monotonicity of the relationship between dynamic energy consumption and performance events (PMCs) and suggesting the non-existence of a functional relationship as well.

\section{Approach}
Our main goal here is to find non existence of functional relationship. In other words, it means proving that there cannot be a functional relationship.\\
Our first approach to do so is to find two tuples such that n number of performance events are equal to each other within some tolerance, but their dynamic energy consumption is the same. Existence of such a tuple in database will lead us to prove the non-existence of a functional relationship.\\
Second approach is to group n-1 performance events and find dataset where the group size is greater than equal to three. In this approach we first assume here that the dataset provided has dynamic energy consumption equal to a linear combination of its performance events. Then using arithmetic we compute the constant relating to the nth event. This constant will then be used to find the dynamic energy consumption of the third tuple in the group. If this test fails then we would be able to conclude the non-existence linear functional relation.\\
The third approach which follows the second is to operate on the existing records and make new records by simple addition, subtraction with each other or multiplication of records with the constant. And then on these new records, the first approach is run which means finding same dynamic energy consumption and validating the equality of the performance events associated with it. The above three approaches help us to find the non existence of linear relation and also the functional relation. Our approaches are based on finding one irregularity which helps us defy the assumptions that the dataset is linear or have functional relationship. This methodology does not prove or find the relationship. It rather defies the possibility of having one.

\section{Structure of report}
In this report, you must have already seen the motivation behind the project and brief description of the approaches that will be taken to reach the objective.\\
This Introductory chapter is followed by Background research, work done and work plan and the timeline.\\
Background research introduces you to the application of our project in various fields like power supply management, load balancing etc. It talks about Big data as the data for which the existence of relationship has to be found can be very large. It discusses different softwares available in the market which can be used to do the computation explaining the pros and cons of each. It explains the approaches mentioned above with proofs. So that one can be sure about the correctness of the program and understand the result. Performance of these approaches is also been discussed as performance is one of the key when the data is really big.\\
Following Background research is the work done. Work done contains simple observation that one found and analysis which will reflect the work plan. Work plan contains the timeline for the project.